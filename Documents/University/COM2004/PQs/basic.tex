\documentclass{article}

\usepackage[margin=1in]{geometry}

\usepackage{amssymb}
\usepackage{amsmath}

%\usepackage{minipage}

\usepackage{graphicx}
\graphicspath{{images/}}

\usepackage{float}

\title{Practice Questions}
\author{Thomas Lawson}

\begin{document}

\maketitle

\newpage
\section{Sheet 1}

\subsection*{Question 1) X is a continuous random variable representing the time of day at which an event occurs, measured in hours 0 to 24. The event is equally likely to occur at any time.}

\subsection*{(i) Sketch the probability density function, $p(x)$}
\begin{figure}[H]
  \centering
  \includegraphics[width=0.7\textwidth]{0101.png}
  \label{fig:yourlabel}
\end{figure}

\subsection*{(ii) Write out the equation for $p(x)$.} 

We know that $p(x)$ has a value of $\frac{1}{24}$ if and only if $0 \le x \le 24$, otherwise it has a value of 0 (no chance of it happening).

\medbreak
\noindent
With this we can form the following equation:

$$p(x) = 
\begin{cases} 
  \frac{1}{24} & \text{if } 1 \le x \le 24 \\ 
  0 & \text{otherwise} 
\end{cases}$$



\subsubsection*{(iii) What is the probability that the event happening between 8:00 and 12:00?}

This is a 4 hour period. A 4 hour period makes up $\frac{1}{6}$ hours of a day.

\medbreak
\noindent
The event is equally likely to happen at any time of the day and the area under the PDF is 1 $\therefore p(8 \le x \le 12) = 1 \times \frac{1}{6} = \frac{1}{6}$ 

\subsection*{Question 2) A pdf, p(x), has a value that linearly increases from x = 0 to x = a, remains constant from x = a to x = b and then linearly decreases from x = b to x = c, where 0 < a < b < c. p(x) = 0 for x < 0 and for x > c.
The pdf has no discontinuities.}

\subsubsection*{(i) Sketch $p(x)$}

\begin{figure}[H]
  \centering
  \includegraphics[width=0.7\textwidth]{0102.png}
  \label{fig:yourlabel}
\end{figure}

\subsubsection*{(ii) What is the value of p(x) for x = (a + b)/2}

The equation we've been give here is almost that of the area of a trapezium ($A = \frac{a-b}{2}\times h$) except there is no $h$. If you divide both sides by $h$ then that gives us 

$$\frac{A}{h} = \frac{a+b}{2}$$

\medbreak
\noindent
The RHS of this equation is the same as the RHS of the equation of given in the question which means that $x = \frac{A}{h}$

\medbreak
\noindent
$A$ reprents the area of the trapezium, the pdf has the shape of the trapezium and all pdfs have an area under the curve equal to one $\therefore\, A = 1$. 

\medbreak
\noindent
We can now rearrange to get $h$

$$\frac{1}{h} = \frac{a+b}{2}$$
$$1 = \frac{a+b}{2}\times h$$
$$\frac{1}{\frac{a+b}{2}} = h$$
$$\frac{2}{a+b} = h$$

\medbreak
\noindent
I can now substitute the values corresponding to the sketch, not the original equation of the trapezium giving us:

$$\frac{2}{b-a+c} = h$$

\subsubsection*{(iii) What is the probability that $x$ has a value less than $a$}

We can get this by doing $h \times a \times 0.5$ just like when calculating the area of a triangle.

$$h \times a \times \frac{1}{2} = \frac{2}{b-a+c} \times a \times \frac{1}{2} = \frac{1}{b-a+c} \times a = \frac{a}{b-a+c}$$

\subsubsection*{(iv) What is the probability that $x$ has a value greater than $b$}

We can get this by by doing $h \times (c-b) \times 0.5$ just like when calculating the area of a triangle.

$$h \times (c-b) \times 0.5 = \frac{2}{b-a+c} \times (c-b) \times \frac{1}{2} = \frac{c-b}{b-a+c}$$


\subsection*{Question 3) On an alien planet there are two types of tree, type A and type B. The heights of type A have a pdf that
linearly decreases from a maximum at a height of 0 m down to a value of 0 at a height of 10 m. The heights
of type B have a pdf that is 0 at 0 m and linearly increases to a maximum value at 10 m. There are no trees
taller than 10 m. Assume also that there are an equal number of A and Bs on the planet.}

\subsubsection*{(i) Sketch the pdfs of the heights for type A and type B, ie., p(x|A) and p(x|B)}

\begin{figure}[H]  % optional, for floating figures
  \centering
  \includegraphics[width=0.7\textwidth]{graph.png}
  \label{fig:yourlabel}
\end{figure}

\subsubsection*{(ii) What is the probability of a type A tree being between 3 and 4 meters tall?}

We're trying to find the area under the graph for $p(3 \le x \le 4 | \,A)$.

\medbreak
\noindent
This area forms a trapezium, which means we can solve the area using $A = \frac{b+a}{2} \times h_t$. We can easily calculate $h_t$ by doing $4-3$ which gives us $1$.

\medbreak
\noindent
To find the values of $p(x = 4\,|\,A)$ and $p(x = 3\,|\,A)$ we can start by finding the equation for $p(x\,|\,A)$ which is $\frac{1}{5}\times\frac{10 - x}{10}$.  

\medbreak
\noindent
Using this we can get the values for $p(x = 4\,|\,A)$ and $p(x = 3\,|\,A)$.

$$p(x = 4\,|\,A) = \frac{1}{5}\times\frac{10 - 4}{10} = \frac{1}{5} \times \frac{6}{10} = \frac{6}{50}$$
$$p(x = 3\,|\,A) = \frac{1}{5}\times\frac{10 - 3}{10} = \frac{1}{5} \times \frac{7}{10} = \frac{7}{50}$$

\medbreak
\noindent
Now we can find the probability of tree type $A$ being between 3 and 4

$$\frac{\frac{7}{50} + \frac{6}{50}}{2} \times 1 = \frac{\frac{13}{50}}{2} = \frac{13}{100}$$


\subsubsection*{(iii) A tree is found and observed to be 6 meters tall. What is the probability that it is of type A?}

We can deduce that the equation for $p(x|B)$ is $\frac{1}{5} \times \frac{x}{10}$.

\medbreak
\noindent
Since there are the same number of type A and type B trees, we can divide the probability of $p(x=6|A)$ by $p(x=6|A) + p(x=6|B)$ to get our answer.

$$\frac{p(x=6|A)}{p(x=6|A)+p(x=6|B)} = \frac{\frac{1}{5}\times\frac{10-6}{10}}{\frac{1}{5}\times \frac{10-6}{10}+\frac{1}{5}\times \frac{6}{10}} = \frac{\frac{10-6}{10}}{\frac{10-6}{10}+\frac{6}{10}} = \frac{\frac{4}{10}}{\frac{4}{10} + \frac{6}{10}} = \frac{4}{4+6} = \frac{4}{10}$$

\subsubsection*{(iv) Sketch the posterior probability p(A|x) as a function of x.}


\begin{figure}[H]  % keeps it in place
  \centering
  \begin{minipage}{0.45\textwidth}  % Text takes 45% of width
    \begin{center}Pre-Sketch Notes\end{center}
    $p(A|x)$ must have a value of 0.5 when $x=0$, as at that point there should be no type B trees.  
    It should be 0 when $x=10$. The line should start at 0 and end at 10.

    \medbreak
    \noindent
    The sketch should look identical to the one done of $p(x|A)$ earlier in the question.
  \end{minipage}%
  \hfill
  \begin{minipage}{0.45\textwidth}  % Image takes 45% of width
    \centering
    \includegraphics[width=\textwidth]{0103iv.png}
  \end{minipage}
\end{figure}

\subsubsection*{(v) Using Bayes' decision rule, what would be the threshold height that a classifier would use to seperate Type A and Type B?}

I'm looking to find a value for $x$ where we start getting $p(A|x) < p(B|x)$ instead of $p(A|x) > p(B|x)$.

\medbreak
\noindent
We haven't calculated $p(A|x)$ nor $p(B|x)$ yet but we can tell that they hold that same values as $p(x|A) and p(x|B)$ respectively

\medbreak
\noindent
Here are $p(A|x)$ and $p(B|x)$ represented by Baye's rule 

$$p(A|x) = \frac{p(x|A) \times p(A)}{P(x)}$$
$$p(B|x) = \frac{p(x|B) \times p(B)}{P(x)}$$

\medbreak
\noindent
The question is asking me to calculate the threshold height using Bayes' decision rule so I'll set the two as equal to each other and to obtain the value of $x$ where they return the same value which should represent the threshold height.

$$\frac{p(x|A) \times p(A)}{P(x)} = \frac{p(x|B) \times p(B)}{P(x)}$$

\medbreak
\noindent
Cancel out $p(x)$

$$p(x|A) \times p(A) = p(x|B) \times p(B)$$

\medbreak
\noindent
Cancel out $p(A)$ and $p(B)$ as there are the same amount of each $\therefore$ they have the same value.

$$p(x|A) = p(x|B)$$

\medbreak
\noindent
Substitute in our equations from earlier

$$\frac{1}{5} \times \frac{10-x}{10} = \frac{1}{5} \times \frac{x}{10}$$

\medbreak
\noindent
Now we can just simplify

$$\frac{10-x}{10} = \frac{x}{10}$$
$$\frac{10-x}{10} = \frac{x}{10}$$
$$10-x = x$$
$$10 = 2x$$
$$5 = x$$

\medbreak
\noindent
Now we know that the threshold height is $5$, if $x < 5$ it can be classied as type $A$, otherwise as $B$.

\subsubsection*{(vi) What would be the probability of a classification error for this classifier?}

\underline{First attempt (Wrong)}

\medbreak
\noindent
I'll calculate this by getting the area under each of the two PDFs on each side of the threshold height. I will then use to to calculate probability of each side being incorrectly labelled.

\medbreak
\noindent
I'll start by calculating area under $p(A|x \le 5)$

$$\frac{\frac{1}{5}+ \frac{1}{5} \times \frac{10-5}{10}}{2} \times 5 = \frac{\frac{1}{5}+ \frac{1}{5} \times \frac{1}{2}}{2} \times 5 = \frac{\frac{1}{5}+ \frac{1}{10}}{2} \times 5 = \frac{1+\frac{1}{5}}{2} = \frac{\frac{6}{5}}{2} = \frac{6}{10} $$

\medbreak
\noindent
This means that $p(A|x \ge 5) = 0.4$

\medbreak
\noindent
The graph mirrors at $x=5$ so we can set inverse values for $B$, $p(B|x \le 5) = 0.4$ and $p(B|x \le 5) = 0.6$

Now we can calculate how often the classifier will be wrong. 

$$(\frac{0.4}{0.4+0.6}\times 0.5)\times 2 = \frac{4}{10}$$

\medbreak
\noindent
\underline{Second attempt (This one is right)}

\medbreak
\noindent
I'll calculate this by getting the area under each of the two PDFs on each side of the threshold height. I will then use to to calculate probability of each side being incorrectly labelled.

\medbreak
\noindent
I'll start by calculating area under $p(A|x \le 5)$

$$\frac{\frac{1}{5}+ \frac{1}{5} \times \frac{10-5}{10}}{2} \times 5 = \frac{\frac{1}{5}+ \frac{1}{5} \times \frac{1}{2}}{2} \times 5 = \frac{\frac{1}{5}+ \frac{1}{10}}{2} \times 5 =\frac{\frac{3}{10}}{2} \times 5 = \frac{3}{20} \times 5 = \frac{15}{20}=\frac{3}{4}$$ 

\medbreak
\noindent
This means that $p(A|x \ge 5) = \frac{1}{4}$

\medbreak
\noindent
The graph mirrors at $x=5$ so we can set inverse values for $B$, $p(B|x \le 5) = \frac{1}{4}$ and $p(B|x \le 5) = \frac{3}{4}$

\medbreak
\noindent
Now we can calculate how often the classifier will be wrong. 

$$(\frac{\frac{1}{4}}{\frac{1}{4}+\frac{3}{4}}\times 0.5)\times 2 = \frac{1}{4}$$

\subsubsection*{(vii) If it is now discovered that there are in fact twice as many type B trees as type A trees, i.e. the prior
probabilities are not equal. What would be the new best decision threshold?}

To get the new value for the decision threshold, we can just continue from midway part $v$ of this question at the point where the prior probabilities were cancelled out except we don't cancel out for this questions as $p(A)$ and $p(B)$ no longer hold the same value.

$$p(x|A) \times p(A) = p(x|B) \times p(B)$$

\medbreak
\noindent
We can now substitue in

$$\frac{1}{5} \times \frac{10-x}{10} \times \frac{1}{3} = \frac{1}{5} \times \frac{x}{10} \times \frac{2}{3}$$

\medbreak
\noindent
Multiply both sides by 15

$$\frac{10-x}{10} = \frac{2x}{10}$$

\medbreak
\noindent
Multiply both sides by 10

$$10-x = 2x$$

\medbreak
\noindent
Add $x$ to both sides

$$10 = 3x$$

\medbreak
\noindent
We get a threshold value for $x$ by dividing both sides by 3

$$\frac{10}{3} = x$$

\subsubsection*{(viii) What would be the probability of classification error in this unequal prior scenario?}

With this new  decision threshold, the classifier should classify any tree below $\frac{10}{3}$ meters as tree type A, otherwise it will class it as type B.

\medbreak
\noindent
I'll start by calculating the probability of $p(x \le \frac{10}{3} | A)$


$$p(x \le \frac{10}{3}|A) = \frac{\frac{1}{5} +  \frac{1}{5} \times \frac{10-\frac{10}{3}}{10} }{2}\times \frac{10}{3} = \frac{\frac{10}{5}+\frac{1}{5}\times (10 -\frac{10}{3})}{6} = \frac{\frac{10}{5}+\frac{10}{5}-\frac{10}{15}}{6} = \frac{\frac{60}{15}-\frac{10}{15}}{6} = \frac{\frac{50}{15}}{6} = \frac{50}{90} = \frac{5}{9}$$


\medbreak
\noindent
From this we can deduce that $p(x \ge \frac{10}{3} | A) = \frac{4}{9}$

\medbreak
\noindent
Now we calculate the probability of $p(x \le \frac{10}{3} | B)$

$$p(x \le \frac{10}{3}|B) = \frac{0 +  \frac{1}{5} \times \frac{\frac{10}{3}}{10}}{2}\times \frac{10}{3} = \frac{ \frac{1}{5} \times \frac{1}{3}}{2}\times \frac{10}{3} = \frac{ \frac{1}{15}}{2}\times \frac{10}{3} = \frac{\frac{10}{15}}{6} = \frac{\frac{2}{3}}{6} = \frac{2}{18} = \frac{1}{9}$$

\medbreak
\noindent
From this we can deduce that $p(x \ge \frac{10}{3} | B) = \frac{8}{9}$

\medbreak
\noindent
Now we can find the likelihood of $x \le \frac{10}{3}$ being misslabeled.

$$\frac{\frac{1}{9} \times 2}{\frac{5}{9} + \frac{1}{9} \times 2} = \frac{\frac{2}{9}}{\frac{5}{9} + \frac{2}{9}} = \frac{2}{7}$$

\medbreak
\noindent
Now we can find the likelihood of $x \ge \frac{10}{3}$ being misslabeled.

$$\frac{\frac{4}{9}}{\frac{4}{9} + \frac{8}{9} \times 2} = \frac{\frac{4}{9}}{\frac{4}{9} + \frac{16}{9}} = \frac{4}{20} = \frac{1}{5}$$

\medbreak
\noindent
Now I will find the proportion of trees smaller than $\frac{10}{3}$

$$p(x \le \frac{10}{3}) = \frac{\frac{5}{9} + \frac{1}{9}\times 2}{\frac{5}{9} + \frac{1}{9}\times 2 + \frac{4}{9} + \frac{8}{9} \times 2} = \frac{\frac{7}{9}}{\frac{7}{9} + \frac{20}{9}} = \frac{7}{27}$$

We can deduce now that $p(x \le \frac{10}{3}) = \frac{20}{27}$

Now we can finally calculate the missclassification probability

$$p(x_\text{missclassified}) = $$

\begin{align*}
  p(x_{misslabeled}) &= p(x_{misslabeled}|x \le \frac{10}{3}) \times p(x \le \frac{10}{3}) + p(x_{misslabeled}|x \ge \frac{10}{3}) \times p(x \ge \frac{10}{3})\\
  &= \frac{2}{7} \times \frac{7}{27} + \frac{1}{5} \times \frac{20}{27}   \\
  &= \frac{14}{189} + \frac{20}{135} \\
  &= \frac{14}{189} + \frac{20}{135} \\
  &= \frac{210}{845} \\
  &= \frac{2}{9} \\
\end{align*}

\subsubsection*{(ix) What is the average height of the type A trees? Of the type B trees?}

We will use the following equation to calculate the average heights (I had to google this):

$$\langle X \rangle = \int_{a}^{b} x \, p(x) \, dx$$

\medbreak
\noindent
First we solve the average height for type A trees

\begin{align*}
  \int_{0}^{10} x \times (\frac{1}{5} \times \frac{10-x}{10}) \, dx &= \int_{0}^{10} x \times (\frac{10-x}{50}) \, dx  \\
  &= \int_{0}^{10} \frac{x(10-x)}{50} \, dx  \\
  &= \int_{0}^{10} \frac{10x-x^2)}{50} \, dx  \\
  &= \int_{0}^{10} \frac{10x}{50}-\frac{x^2}{50} \, dx  \\
  &= \frac{1}{50} \int_{0}^{10} 10x-x^2 \, dx  \\
  &= \frac{1}{50} [5x^2-\frac{x^3}{3}]^{10}_0   \\
  &= \frac{1}{50} (5(10)^2-\frac{10^3}{3})   \\
  &= \frac{1}{50} (5 \times 100 -\frac{1000}{3})   \\
  &= \frac{1}{50} (500 -\frac{1000}{3})   \\
  &= \frac{1}{50} (\frac{500}{3})   \\
  &= \frac{500}{150}   \\
  &= \frac{50}{15}   \\
  &= \frac{10}{3}   \\
\end{align*}

\medbreak
\noindent
So the average height for type A trees is $\frac{10}{3}$

\medbreak
\noindent
Now we can use the same technique to calculate the average height for type B trees

\begin{align*}
  \int_{0}^{10} x \times (\frac{1}{5} \times \frac{x}{10}) \, dx &= \int_{0}^{10} \frac{x}{5} \times \frac{x}{10} \, dx  \\
  &= \int_{0}^{10} \frac{x^2}{50}\, dx  \\
  &= \frac{1}{50} \int_{0}^{10} x^2\, dx  \\
  &= \frac{1}{50} [\frac{x^3}{3}]^{10}_0  \\
  &= \frac{1}{50} \times \frac{10^3}{3}  \\
  &= \frac{10^3}{150} \\
  &= \frac{1000}{150} \\
  &= \frac{100}{15} \\
  &= \frac{20}{3} \\
\end{align*}


\newpage
\section*{Sheet 2}

\subsection*{Question 1 - Vector operations}

Consider the vectors $\vec{x}_1$ and $\vec{x}_2$ given below,
\[
\vec{x}_1 = (3,2,1)^T, \quad \vec{x}_2 = (-2,1,0)^T
\]

For each part below, first work out the answers by hand and then check your answers using Python and numpy.

\begin{enumerate}
  \renewcommand{\labelenumi}{(\roman{enumi})}

  \item The lengths of $\vec{x}_1$ and $\vec{x}_2$.

  We can find the magnitude of these two vectors by taking the square root
  of the sum of the squares of each element.

  \[
  |\vec{x}_1| = \sqrt{3^2 + 2^2 + 1^2} = \sqrt{9 + 4 + 1} = \sqrt{14}
  \]
  \[
  |\vec{x}_2| = \sqrt{(-2)^2 + 1^2 + 0^2} = \sqrt{4 + 1 + 0} = \sqrt{5}
  \]

  \item The $\ell_1$-norm distance from $\vec{x}_1$ to $\vec{x}_2$.

  \[
  \|\vec{x}_1 - \vec{x}_2\|_1 = \sum_{i=1}^{3} |x_{1i} - x_{2i}|
  = |3 - (-2)| + |2 - 1| + |1 - 0| = 7
  \]

  \item The $\ell_2$-norm distance from $\vec{x}_1$ to $\vec{x}_2$.

  \[
  \|\vec{x}_1 - \vec{x}_2\|_2 =
  \left( \sum_{i=1}^{3} (x_{1i} - x_{2i})^2 \right)^{1/2}
  = \sqrt{(3 - (-2))^2 + (2 - 1)^2 + (1 - 0)^2}
  = \sqrt{27}
  \]

  \item The \textit{inner} product $\vec{x}_1^T \vec{x}_2$.

  \[
  \vec{x}_1^T \vec{x}_2 = (3)(-2) + (2)(1) + (1)(0) = -4
  \]

  \item The \textit{outer} product $\vec{x}_1 \vec{x}_2^T$.

  \[
  \vec{x}_1 \vec{x}_2^T =
  \begin{bmatrix}
  3 \\ 2 \\ 1
  \end{bmatrix}
  \begin{bmatrix}
  -2 & 1 & 0
  \end{bmatrix}
  =
  \begin{bmatrix}
  -6 & 3 & 0 \\
  -4 & 2 & 0 \\
  -2 & 1 & 0
  \end{bmatrix}
  \]
\end{enumerate}


\mathbf{x}_1 \otimes \mathbf{x}_2 
&= 
\mathbf{x}_1 \mathbf{x}_2^T \\[6pt]
&=
\begin{bmatrix}
3 \\ 2 \\ 1
\end{bmatrix}
\begin{bmatrix}
-2 & 1 & 0
\end{bmatrix} \\[6pt]
&=
\begin{bmatrix}
3(-2) & 3(1) & 3(0) \\[4pt]
2(-2) & 2(1) & 2(0) \\[4pt]
1(-2) & 1(1) & 1(0)
\end{bmatrix} \\[6pt]
&=
\begin{bmatrix}
-6 & 3 & 0 \\[4pt]
-4 & 2 & 0 \\[4pt]
-2 & 1 & 0
\end{bmatrix}
\end{aligned}


}$$

    \item Cosine of the angle between $\vec{x}_1$ and $\vec{x}_2$.

      $$\begin{aligned}
\cos \theta 
&= \frac{\mathbf{x}_1 \cdot \mathbf{x}_2}{\|\mathbf{x}_1\| \, \|\mathbf{x}_2\|} \\[6pt]
&= \frac{(3)(-2) + (2)(1) + (1)(0)}%
{\sqrt{3^2 + 2^2 + 1^2} \; \sqrt{(-2)^2 + 1^2 + 0^2}} \\[6pt]
&= \frac{-6 + 2 + 0}{\sqrt{14} \, \sqrt{5}} \\[6pt]
&= \frac{-4}{\sqrt{70}} \\[6pt]
&\approx -0.478
\end{aligned}$$

    \item The projection of $\vec{x}_1$ onto $\vec{x}_2$.
      $$
\frac{\mathbf{x}_1 \cdot \mathbf{x}_2}{\|\mathbf{x}_2\|^2} \, \mathbf{x}_2 \\[6pt]
&= \frac{(3)(-2) + (2)(1) + (1)(0)}{(-2)^2 + 1^2 + 0^2}
\begin{bmatrix} -2 \\ 1 \\ 0 \end{bmatrix} \\[6pt]
&= \frac{-6 + 2 + 0}{5}
\begin{bmatrix} -2 \\ 1 \\ 0 \end{bmatrix} \\[6pt]
&= \frac{-4}{5}
\begin{bmatrix} -2 \\ 1 \\ 0 \end{bmatrix} \\[6pt]
&=
\begin{bmatrix}
\frac{8}{5} \\[4pt]
-\frac{4}{5} \\[4pt]
0
\end{bmatrix}
\end{aligned}$$

    \item The projection of $\vec{x}_2$ onto $\vec{x}_1$.
\end{enumerate}
$$\begin{aligned}
\frac{\mathbf{x}_1 \cdot \mathbf{x}_2}{\|\mathbf{x}_1\|^2} \, \mathbf{x}_1
&= \frac{-4}{3^2 + 2^2 + 1^2}
\begin{bmatrix} 3 \\ 2 \\ 1 \end{bmatrix} \\[6pt]
&= \frac{-4}{14}
\begin{bmatrix} 3 \\ 2 \\ 1 \end{bmatrix} \\[6pt]
&=
\begin{bmatrix}
-\frac{6}{7} \\[4pt]
-\frac{4}{7} \\[4pt]
-\frac{2}{7}
\end{bmatrix}
\end{aligned}$$

\bigskip
\subsection*{Question 2 - Multivariate data analysis}


Consider the following 5 samples of some 3-dimensional data

\[
\vec{x}_1 = (100, 200, 297)^T
\]

\[
\vec{x}_2 = (104, 202, 306)^T
\]

\[
\vec{x}_3 = (96, 198, 300)^T
\]

\[
\vec{x}_4 = (90, 195, 302)^T
\]

\[
\vec{x}_5 = (110, 205, 295)^T
\]

For each part below, first work out the answers by hand and then check your answers using Python and numpy.

\begin{enumerate}
    \renewcommand{\labelenumi}{(\roman{enumi})}
    \item The 3-D sample mean vector
      
      $|\vec{x}|_1 = \frac{100+104+96+90+110}{5} = \frac{500}{5} = 100$

      $|\vec{x}|_2 = \frac{200+202+198+195+205}{5} = \frac{1000}{5} = 200$

      $|\vec{x}|_2 = \frac{297+306+300+302+295}{5} = \frac{1500}{5} = 300$

      $|\vec{x}| = (100, 200, 300)^T$

    \item The 3-D sample variance vector

\begin{align*}
s_{11} &= \frac{1}{N-1}\sum_{N}^{1} (x_{i1}-|x_1|)^2 \\
       &= \frac{1}{5-1}\sum_{5}^{1} (x_{i1}-|x_1|)^2 \\
       &= \frac{1}{4} \times (100-100)^2 + (104-100)^2 + (96-100)^2 + (90-100)^2 + (110-100)^2 \\
       &= \frac{1}{4}(0 + 36 + 16 + 100 + 100) \\
       &= \frac{1}{4}(52 + 200) \\
       &= \frac{252}{4} \\
       &= 63
\end{align*}

\begin{align*}
s_{22} &= \frac{1}{N-1}\sum_{N}^{1} (x_{i2}-|x_2|)^2 \\
       &= \frac{1}{5-1}\sum_{5}^{1} (x_{i2}-|x_2|)^2 \\
       &= \frac{1}{4} \times (200-200)^2 + (202-200)^2 + (198-200)^2 + (195-200)^2 + (205-200)^2 \\
       &= \frac{1}{4}(0+4+4+25+25) \\
       &= \frac{58}{4} \\
\end{align*}

\begin{align*}
s_{33} &= \frac{1}{N-1}\sum_{N}^{1} (x_{i3}-|x_3|)^2 \\
       &= \frac{1}{5-1}\sum_{5}^{1} (x_{i3}-|x_3|)^2 \\
       &= \frac{1}{4} \times (297-300)^2 + (306-300)^2 + (300-300)^2 + (302-300)^2 + (295-300)^2 \\
       &= \frac{1}{4}(9+36+0+4+25) \\
       &= \frac{1}{4}(74) \\
       &= \frac{74}{4}
\end{align*}

    This means that we have the covariance vector $x_{variance}= (\frac{252}{4}, \frac{58}{4}, \frac{74}{4})^T$

    \item The 3-by-3 sample covariance matrix

      There are another 6 elements in the $s$ covariance matrix to calculate, however, I only need to calculate another 3 as covariance matrices are symmetrical along the diagonal.

\begin{align*}
  s_{12} &= \frac{1}{N-1}\sum_{N}^{1} (x_{i1}-|x_1|)(x_{i2}-|x_2|) \\
          &= \frac{1}{4}\sum_{N}^{1} (x_{i1}-|x_1|)(x_{i2}-|x_2|) \\
          &= \frac{1}{4}\times(0 \times 0 + 4 \times 2 + (-4)\times(-2) + (-10) \times (-5)+ 10 \times 5) \\
          &= \frac{1}{4}\times(8 + 8 + 50 + 50) \\
          &= \frac{116}{4}\\
\end{align*}

    Now we do the same for $s_{13}$
\begin{align*}
  s_{12} &= \frac{1}{N-1}\sum_{N}^{1} (x_{i1}-|x_1|)(x_{i3}-|x_3|) \\
          &= \frac{1}{4}\sum_{N}^{1} (x_{i1}-|x_1|)(x_{i3}-|x_3|) \\
          &= \frac{1}{4}\times 0 \times (-3) + 4 \times 6 + (-4)\times 0 + (-10) \times 2+ 10 \times (-5) \\
          &= \frac{1}{4}\times(24-20-50) \\
          &= -\frac{46}{4}\\
\end{align*}

Finally for $x_{23}$
\begin{align*}
  s_{12} &= \frac{1}{N-1}\sum_{N}^{1} (x_{i2}-|x_2|)(x_{i3}-|x_3|) \\
          &= \frac{1}{4}\sum_{N}^{1} (x_{i2}-|x_2|)(x_{i3}-|x_3|) \\
          &= \frac{1}{4}\times (0 \times (-3) + 2 \times 6 + (-2)\times 0 + (-5) \times 2+ 5 \times (-5)) \\
          &= \frac{1}{4}\times(0+12-10-25) \\
          &= -\frac{23}{4}\\
\end{align*}

    With all these covariances calculated, this gives us all the values we need to form the covariance matrix

    $$ S = 
\begin{bmatrix}
  \frac{252}{4} & \frac{116}{4} & \frac{46}{4} \\
  \frac{116}{4} & \frac{58}{4} & -\frac{23}{4} \\
  \frac{46}{4} & -\frac{23}{4} & \frac{74}{4} \\
\end{bmatrix} = \frac{1}{4}
\begin{bmatrix}
  252 & 116 & \frac{46}{4} \\
  116 & 58 & -\frac{23}{4} \\
  46 & 23 & \frac{74}{4} \\
\end{bmatrix} 
$$

    \item Identify the sample that has greatest $\ell_1$-norm distance from the mean

      We already calculated the mean vector in part (i) which is $(100, 200, 300)^T$

      So now we calculate the $\ell_1$-norm distance for each of the vectors

      $$\begin{aligned}
\|\vec{x}_1 - \bar{x}\|_1 &= |100 - 100| + |200 - 200| + |297 - 300| \\
&= 0 + 0 + 3 = 3 \\[8pt]
\|\vec{x}_2 - \bar{x}\|_1 &= |104 - 100| + |202 - 200| + |306 - 300| \\
&= 4 + 2 + 6 = 12 \\[8pt]
\|\vec{x}_3 - \bar{x}\|_1 &= |96 - 100| + |198 - 200| + |300 - 300| \\
&= 4 + 2 + 0 = 6 \\[8pt]
\|\vec{x}_4 - \bar{x}\|_1 &= |90 - 100| + |195 - 200| + |302 - 300| \\
&= 10 + 5 + 2 = 17 \\[8pt]
\|\vec{x}_5 - \bar{x}\|_1 &= |110 - 100| + |205 - 200| + |295 - 300| \\
&= 10 + 5 + 5 = 20
\end{aligned}
$$

    We can see that the distance is greatest to $x_5$

    \item Identify the sample that has greatest $\ell_2$-norm distance from the mean

    Now we calculate the $\ell_2$ distance for each of the vectors

$$
\|\vec{x}_1 - \bar{x}\|_2 &= \sqrt{(100-100)^2 + (200-200)^2 + (297-300)^2} \\
&= \sqrt{0^2 + 0^2 + (-3)^2} \\
&= \sqrt{9} = 3 \\[6pt]
$$
$$
\|\vec{x}_2 - \bar{x}\|_2 &= \sqrt{(104-100)^2 + (202-200)^2 + (306-300)^2} \\
&= \sqrt{4^2 + 2^2 + 6^2} \\
&= \sqrt{16 + 4 + 36} \\
&= \sqrt{56} \approx 7.483 \\[6pt]
$$

$$
\|\vec{x}_3 - \bar{x}\|_2 &= \sqrt{(96-100)^2 + (198-200)^2 + (300-300)^2} \\
&= \sqrt{(-4)^2 + (-2)^2 + 0^2} \\
&= \sqrt{16 + 4 + 0} \\
&= \sqrt{20} \approx 4.472 \\[6pt]
$$

$$
\|\vec{x}_4 - \bar{x}\|_2 &= \sqrt{(90-100)^2 + (195-200)^2 + (302-300)^2} \\
&= \sqrt{(-10)^2 + (-5)^2 + 2^2} \\
&= \sqrt{100 + 25 + 4} \\
&= \sqrt{129} \approx 11.357 \\[6pt]
$$
$$
\|\vec{x}_5 - \bar{x}\|_2 &= \sqrt{(110-100)^2 + (205-200)^2 + (295-300)^2} \\
&= \sqrt{10^2 + 5^2 + (-5)^2} \\
&= \sqrt{100 + 25 + 25} \\
&= \sqrt{150} \approx 12.247
$$

$x_5$ has the $\ell_2$ distance to the mean.
\end{enumerate}



\bigskip
\subsection*{Question 3 - Expected values}

The continuous random variable $X$ has a uniform distribution between the limits of $a$ and $b$.

\begin{enumerate}
    \renewcommand{\labelenumi}{(\roman{enumi})}
    \item Evaluate the expected value, $E(X)$.
    \item Using the fact that variance is $E((X - E(X))^2)$, show that the variance of $X$ is $\frac{1}{12} (a - b)^2$.
\end{enumerate}

\bigskip
\subsection*{Question 4 - Parameter Estimation -- Uniform distribution}

Consider a uniform distribution between 0 and $b$ where $b$ is unknown, i.e. $X \sim \mathcal{U}(0,b)$.

\begin{enumerate}
    \renewcommand{\labelenumi}{(\roman{enumi})}
    \item A single sample is observed with a value of 5.0. What is the maximum likelihood (ML) estimate for the parameter $b$?
    \item A series of observations are observed with values $\{4.1, 4.3, 4.6, 5.0\}$. Again, what is the ML estimate for $b$?
    \item After estimating $b$ write down the expected value of $X$. Compare $E(X)$ with the mean value of the samples. Are they the same?
    \item In Clockworkville buses run at precise fixed intervals 24 hours a day, i.e. a bus will arrive precisely on time every $x$ minutes. A man, knowing this fact, but with no knowledge of the bus schedule or the interval between the buses goes to catch a bus. He has to wait exactly 5 minutes before a bus arrives. The next day a friend asks him, ``How often do the buses come?'' What should he reply?
\end{enumerate}

\newpage
\section{Sheet 3}

\subsection*{Question 1}

\begin{enumerate}
    \renewcommand{\labelenumi}{(\roman{enumi})}
    \item The 3-D sample mean vector
\end{enumerate}




\end{document} 
